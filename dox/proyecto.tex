% Definimos el estilo del documento
\documentclass[12pt,a4paper,spanish]{book}
% Utilizamos el paquete para utilizar espa�ol
\usepackage[spanish]{babel}
% Utilizamos el paquete para gestionar acentos
\usepackage[latin1]{inputenc}
%Utilizamos el paquete para incorporar graficos postcript
\usepackage[dvips,final]{epsfig}
% Definimos la zona de la pagina ocupada por el texto
\oddsidemargin 0.1cm \headsep 0.5cm \textwidth=15.5cm
\textheight=22cm
%Empieza el documento
% Para que numere hasta subsubsection
\setcounter{secnumdepth}{3}
\setcounter{tocdepth}{3}
\begin{document}
% Definimos titulo, autor, fecha, generamos titulo e indice de
\title{Proyecto final de carrera}
\author{Diego Garc�a Valverde}
\date{Julio 2008}
\maketitle
\tableofcontents
% Definimos una primera pagina para los agradecimientos
\newpage
\thispagestyle{empty}
\section*{Agradecimientos}
Aqu� pongo los agradecimientos
% Empezamos cap�tulos
\chapter{Introducci\'on}
La introducci�n es lo primero que se lee, pero habitualmente lo
�ltimo que se escribe. Pues su redacci�n
depende de c�mo se hayan escrito todas las otras secciones.
Normalmente la introducci�n incluye una
descripci�n muy general del proyecto y termina con un desglose del
contenido de la memoria.
\section{Objetivo}
un poco de historia
\chapter{Recursos}
\section{Hardware}
Descripci�n del estado actual del tema con referencia a trabajos
anteriores en el caso de proyectos que
sean continuaci�n o relacionados con otros proyectos.
\subsection{Placa base}
\subsection{Monitor t�ctil}
\section{Software}
Metodolog�a a utilizar para el desarrollo del proyecto,
herramientas de an�lisis, etc..
\subsection{SO Debian GNU/Linux}
\subsection{Drivers}
\subsubsection{Gr�fica VIA}
\subsubsection{Monitor touchscreen}
\subsection{Librer�as gr�ficas}
\subsubsection{SDL}
\subsubsection{SDL-gfx}
\subsection{Librer�a sockets}
\subsubsection{SolarSockets++}
\chapter{GUI}
Explicar la estructura de la interfaz
\section{Estructura visual}
\section{Multihilo}
\subsection{Gestor Mapa}
\subsection{Gestor Sensores}
\section{Comunicaci�n}
\subsection{Socket Mapa}
\subsection{Socket Sensores}
\chapter{Plan de trabajo y temporizaci�n}
Desarrollo del plan de trabajo desglosado en etapas, con una
estimaci�n en cada etapa del tiempo de ejecuci�n
\begin{itemize}
\item Etapa 1 de Mi Proyecto
\item Etapa 2 de Mi Proyecto
\item .................
\end{itemize}
\chapter{Resultados y conclusiones}
Desarrollo de los resultados y conclusiones del proyecto. Se debe
intentar resaltar el inter�s del proyecto
y la calidad del trabajo realizado. Ha llegado el momento de
"vender" nuestro trabajo. Se deben incluir aspectos como:
\begin{itemize}
\item Calidad, dificultad y amplitud del trabajo desarrollado que
justifique el tiempo de dedicaci�n al proyecto.
\item Aspectos integradores de las disciplinas de la titulaci�n de
Ingeniero en Inform�tica.
\item Impacto social. Utilidad del proyecto en el �mbito social
\item Facilidad de utilizaci�n de los resultados del proyecto por
terceras personas.
\item Publicidad de los resultados del proyecto a trav�s de p�ginas
web, etc.. Cuando de los resultados del proyecto
se derive un prototipo o programa de utilizaci�n se debe poner
a disposici�n del p�blico en general una versi�n
de demostraci�n de dicho prototipo.
\item Cualquier otro m�rito.
\end{itemize}
\chapter{Trabajo Futuro}
Continuidad del trabajo realizado a trav�s de una implementaci�n,
o utilidad real del proyecto,
o a trav�s de otros proyectos fin de carrera.
% EMPIEZAN LOS APENDICES DEL PROYECTO
\appendix
\chapter{Manual de usuario}
En el caso de que el desarrollo (y/o naturaleza del proyecto haya
dado lugar a la creaci�n de manuales
de usuario, habr� que ponerlo aqu�).
% Aqui va la Bibliograf�a utilizada por el proyecto.
\begin{thebibliography}{1}
\bibitem{La86} Leslie Lamport {\em LaTex : A document Preparation
System}. Addison-Wesley, 1986.
\bibitem{Ro93} Christian Rolland {\em LaTex guide pratique}. Addison-
Wesley, 1993.
\end{thebibliography}

% Termina el documento
\end{document}
